\section{Interactive Editing}

By now, we have seen several examples of how \Idris{}' dependent type system can give extra confidence in a function's correctness by giving a more precise description of its intended behaviour in its \emph{type}.
We have also seen an example of how the type system can help with EDSL development by allowing a programmer to describe the type system of an object language.
However, precise types give us more than verification of programs --- we can also exploit types to help write programs which are \emph{correct by construction}.

The \Idris{} REPL provides several commands for inspecting and modifying parts of programs, based on their types, such as case splitting on a pattern variable, inspecting the type of a metavariable, and even a basic proof search mechanism.
In this section, we explain how these features can be exploited by a text editor, and specifically how to do so in Vim\footnote{\url{https://github.com/idris-hackers/idris-vim}}. 
An interactive mode for
Emacs\footnote{\url{https://github.com/idris-hackers/idris-emacs}} is also available.

\subsection{Editing at the REPL}

The REPL provides a number of commands, which we will describe shortly, which generate new program fragments based on the currently loaded module. These take the general form

\begin{code}
:command [line number] [name]
\end{code}

\noindent
That is, each command acts on a specific source line, at a specific name, and outputs a new program fragment. Each command has an alternative form, which \emph{updates} the source file in-place:

\begin{code}
:command! [line number] [name]
\end{code}

\noindent
When the REPL is loaded, it also starts a background process which accepts and responds to REPL commands, using \verb!idris --client!.
For example, if we have a REPL running elsewhere, we can execute commands such as:

\begin{lstlisting}[language=bash]
$ idris --client ':t plus'
Prelude.Nat.plus : Nat -> Nat -> Nat
$ idris --client '2+2'
4 : Integer
\end{lstlisting}

\noindent
A text editor can take advantage of this, along with the editing commands, in order to provide interactive editing support.

\subsection{Editing Commands}

\subsubsection{:addclause}

The \texttt{:addclause n f} command (abbreviated \texttt{:ac n f}) creates a template definition for the function named \texttt{f} declared on line \texttt{n}.
For example, if the code beginning on line 94 contains:

\begin{code}
vzipWith : (a -> b -> c) -> 
           Vect n a -> Vect n b -> Vect n c
\end{code}

\noindent
then \texttt{:ac 94 vzipWith} will give:

\begin{code}
vzipWith f xs ys = ?vzipWith_rhs
\end{code}

\noindent
The names are chosen according to hints which may be given by a programmer, and then made unique by the machine by adding a digit if necessary.
Hints can be given as follows:

\begin{code}
%name Vect xs, ys, zs, ws
\end{code}

\noindent
This declares that any names generated for types in the \texttt{Vect} family should be chosen in the order \texttt{xs}, \texttt{ys}, \texttt{zs}, \texttt{ws}.

\subsubsection{:casesplit}

The \texttt{:casesplit n x} command, abbreviated \texttt{:cs n x}, splits the pattern variable \texttt{x} on line \texttt{n} into the various pattern forms it may take, removing any cases which are impossible due to unification errors.
For example, if the code beginning on line 94 is:

\begin{code}
vzipWith : (a -> b -> c) -> 
           Vect n a -> Vect n b -> Vect n c
vzipWith f xs ys = ?vzipWith_rhs
\end{code}

\noindent
then \texttt{:cs 96 xs} will give:

\begin{code}
vzipWith f [] ys = ?vzipWith_rhs_1
vzipWith f (x :: xs) ys = ?vzipWith_rhs_2
\end{code}

\noindent
That is, the pattern variable \texttt{xs} has been split into the two possible cases \texttt{[]} and \texttt{x :: xs}. Again, the names are chosen according to the same heuristic.
If we update the file (using \texttt{:cs!}) then case split on \texttt{ys} on the same line, we get:

\begin{code}
vzipWith f [] [] = ?vzipWith_rhs_3
\end{code}

\noindent 
That is, the pattern variable \texttt{ys} has been split into one case \texttt{[]}, \Idris{} having noticed that the other possible case \texttt{y :: ys} would lead to a unification error.

\subsubsection{:addmissing}

The \texttt{:addmissing n f} command, abbreviated \texttt{:am n f}, adds the clauses which are required to make the function \texttt{f} on line \texttt{n} cover all inputs. 
For example, if the code beginning on line 94 is\ldots

\begin{code}
vzipWith : (a -> b -> c) -> 
           Vect n a -> Vect n b -> Vect n c
vzipWith f [] [] = ?vzipWith_rhs_1
\end{code}

\noindent
then \texttt{:am 96 vzipWith} gives:

\begin{code}
vzipWith f (x :: xs) (y :: ys) = ?vzipWith_rhs_2
\end{code}

\noindent
That is, it notices that there are no cases for non-empty vectors, generates the required clauses, and eliminates the clauses which would lead to unification errors.

\subsubsection{:proofsearch}

The \texttt{:proofsearch n f} command, abbreviated \texttt{:ps n f}, attempts to find a value for the metavariable \texttt{f} on line \texttt{n} by proof search, trying values of local variables, recursive calls and constructors of the required family.
Optionally, it can take a list of \emph{hints}, which are functions it can try applying to solve the metavariable.
%
For example, if the code beginning on line 94 is:

\begin{code}
vzipWith : (a -> b -> c) -> 
           Vect n a -> Vect n b -> Vect n c
vzipWith f [] [] = ?vzipWith_rhs_1
vzipWith f (x :: xs) (y :: ys) = ?vzipWith_rhs_2
\end{code}

\noindent
then \texttt{:ps 96 vzipWith\_rhs\_1} will give

\begin{code}
[]
\end{code}

\noindent
This works because it is searching for a \texttt{Vect} of length 0, of which the empty vector is the only possibiliy.
Similarly, and perhaps surprisingly, there is only one possibility if we try to solve \texttt{:ps 97 vzipWith\_rhs\_2}:

\begin{code}
f x y :: (vzipWith f xs ys)
\end{code}

\noindent
This works because \texttt{vzipWith} has a precise enough type:
The resulting vector has to be non-empty (a \texttt{::}); the first element must have type \texttt{c} and the only way to get this is to apply \texttt{f} to \texttt{x} and \texttt{y}; finally, the tail of the vector can only be built recursively.

\subsubsection{:makewith}

The \texttt{:makewith n f} command, abbreviated \texttt{:mw n f}, adds a \texttt{with} to a pattern clause.
For example, recall \texttt{parity}.
If line 10 is:

\begin{code}
parity (S k) = ?parity_rhs
\end{code}

\noindent
then \texttt{:mw 10 parity} will give:

\begin{code}
parity (S k) with (_)
  parity (S k) | with_pat = ?parity_rhs
\end{code}

\noindent
If we then fill in the placeholder \verb!_! with \verb!parity k! and case split on \verb!with_pat! using \verb!:cs 11 with_pat! we get the following patterns:

\begin{code}
  parity (S (plus n n)) | even = ?parity_rhs_1
  parity (S (S (plus n n))) | odd = ?parity_rhs_2
\end{code}

\noindent
Note that case splitting has normalised the patterns here (giving \texttt{plus} rather than \texttt{+}).
In any case, we see that using interactive editing significantly simplifies the implementation of dependent pattern matching by showing a programmer exactly what the valid patterns are.

\subsection{Interactive Editing in Vim}

The editor mode for Vim provides syntax highlighting, indentation and interactive editing support using the commands described above.
Interactive editing is achieved using the following editor commands, each of which update the buffer directly:

\begin{itemize}
\item \verb!\d! adds a template definition for the name declared on the current line (using \texttt{:addclause}).
\item \verb!\c! case splits the variable at the cursor (using \texttt{:casesplit}).
\item \verb!\m! adds the missing cases for the name at the cursor (using \texttt{:addmissing}).
\item \verb!\w! adds a \texttt{with} clause (using \texttt{:makewith}). 
\item \verb!\o! invokes a proof search to solve the metavariable under the cursor (using \texttt{:proofsearch}).
\item \verb!\p! invokes a proof search with additional hints to solve the metavariable under the cursor (using \texttt{:proofsearch}).
\end{itemize}

\noindent
Corresponding commands are also available in the Emacs mode.
Support for other editors can be added in a relatively straighforward manner by using \verb!idris --client!.


