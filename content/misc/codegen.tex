\subsection{JavaScript Target}

\Idris{} is capable of producing \emph{JavaScript} code that can be run in a browser as well as in the \emph{NodeJS} environment or alike.
One can use the FFI to communicate with the \emph{JavaScript} ecosystem.

\subsubsection*{Code Generation}

\noindent
Code generation is split into two separate targets.
To generate code that is tailored for running in the browser issue the following command:

\begin{lstlisting}[style=stdout]
$ idris --codegen javascript hello.idr -o hello.js
\end{lstlisting}

\noindent
The resulting file can be embedded into your HTML just like any other \emph{JavaScript} code.

\noindent
Generating code for \emph{NodeJS} is slightly different.
\Idris{} outputs a \emph{JavaScript} file that can be directly executed via \texttt{node}.

\begin{lstlisting}[style=stdout]
$ idris --codegen node hello.idr -o hello
$ ./hello
Hello world
\end{lstlisting}

\noindent
Take into consideration that the \emph{JavaScript} code generator is using \texttt{console.log} to write text to \texttt{stdout}, this means that it will automatically add a newline to the end of each string.
This behaviour does not show up in the \emph{NodeJS} code generator.

\subsubsection*{Using the FFI}

\noindent
To write a useful application we need to communicate with the outside world.
Maybe we want to manipulate the DOM or send an Ajax request.
For this task we can use the FFI.
Since most \emph{JavaScript} APIs demand callbacks we need to extend the FFI so we can pass functions as arguments.

\noindent
The \emph{JavaScript} FFI works a little bit differently than the regular FFI.
It uses positional arguments to directly insert our arguments into a piece of \emph{JavaScript} code.

\noindent
One could use the primitive addition of \emph{JavaScript} like so:

\begin{code}
module Main

primPlus : Int -> Int -> IO Int
primPlus a b = mkForeign (FFun "%0 + %1" [FInt, FInt] FInt) a b

main : IO ()
main = do
  a <- primPlus 1 1
  b <- primPlus 1 2
  print (a, b)
\end{code}

\noindent
Notice that the \texttt{\%n} notation qualifies the position of the \texttt{n}-th argument given to our foreign function starting from 0.
When you need a percent sign rather than a position simply use \texttt{\%\%} instead.

\noindent
Passing functions to a foreign function is very similar.
Let's assume that we want to call the following function from the \emph{JavaScript} world:

\begin{code}[language=c]
function twice(f, x) {
  return f(f(x));
}
\end{code}

\noindent
We obviously need to pass a function \texttt{f} here (we can infer it from the way we use \texttt{f} in \texttt{twice}, it would be more obvious if \emph{JavaScript} had types).

\noindent
The \emph{JavaScript} FFI is able to understand functions as arguments when you give it something of type \texttt{FFunction}.
The following example code calls \texttt{twice} in \emph{JavaScript} and returns the result to our \Idris{} program:

\begin{code}
module Main

twice : (Int -> Int) -> Int -> IO Int
twice f x = mkForeign (
  FFun "twice(%0,%1)" [FFunction FInt FInt, FInt] FInt
) f x

main : IO ()
main = do
  a <- twice (+1) 1
  print a
\end{code}

\noindent
The program outputs \texttt{3}, just like we expected.

\subsubsection*{Including external \emph{JavaScript} files}

\noindent
Whenever one is working with \emph{JavaScript} one might want to include
external libraries or just some functions that she or he wants to call via
FFI which are stored in external files. The \emph{JavaScript} and \emph{NodeJS} code generators
understand the \texttt{\%include} directive. Keep in mind that \emph{JavaScript} and \emph{NodeJS}
are handled as different code generators, therefore you will have to state which one you want to target.
This means that you can include different files for \emph{JavaScript} and \emph{NodeJS} in the same
\Idris{} source file.

\noindent
So whenever you want to add an external \emph{JavaScript} file
you can do this like so:

\noindent
For \emph{NodeJS}:

\begin{code}
  %include Node "path/to/external.js"
\end{code}

\noindent
And for use in the browser:

\begin{code}
  %include JavaScript "path/to/external.js"
\end{code}

\noindent
The given files will be added to the top of the generated code.

\subsubsection*{Shrinking down generated \emph{JavaScript}}

\Idris{} can produce very big chunks of \emph{JavaScript} code.
However, the generated code can be minified using the \texttt{closure-compiler} from Google.
Any other minifier is also suitable but \texttt{closure-compiler} offers advanced compilation that does some aggressive inlining and code elimination.
\Idris{} can take full advantage of this compilation mode and it's highly recommended to use it when shipping a \emph{JavaScript} application written in \Idris{}.

