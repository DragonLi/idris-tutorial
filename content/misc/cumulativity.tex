\subsection{Cumulativity}

Since values can appear in types and \emph{vice versa}, it is natural that types themselves have types.
For example:

\begin{lstlisting}
*universe> :t Nat
Nat : Type
*universe> :t Vect
Vect : Nat -> Type -> Type
\end{lstlisting} 

\noindent
But what about the type of \texttt{Type}? If we ask \Idris{} it reports

\begin{lstlisting}
*universe> :t Type
Type : Type 1
\end{lstlisting} 

\noindent
If \texttt{Type} were its own type, it would lead to an inconsistency due to Girard's paradox~\cite{girard-thesis}, so internally there is a \emph{hierarchy} of types (or \emph{universes}):

\begin{code}
Type : Type 1 : Type 2 : Type 3 : ...
\end{code} 

\noindent
Universes are \emph{cumulative}, that is, if \texttt{x : Type n} we can also have that \texttt{x : Type m}, as long as \texttt{n < m}. 
The typechecker generates such universe  constraints and reports an error if any inconsistencies are found.
Ordinarily, a
programmer does not need to worry about this, but it does prevent (contrived) programs such as the following:

\begin{code}
myid : (a : Type) -> a -> a
myid _ x = x

idid :  (a : Type) -> a -> a
idid = myid _ myid
\end{code} 

\noindent
The application of \texttt{myid} to itself leads to a cycle in the universe hierarchy --- \texttt{myid}'s first argument is a \texttt{Type}, which cannot be at a lower level than required if it is applied to itself.

%\subsection{Comparison}

%How does \Idris{} compare with other dependently typed languages and proof
%assistants, such as Coq, Agda and Epigram?
